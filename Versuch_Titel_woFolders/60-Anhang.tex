% 6. Anhang

\appendix

% Text

% Anhang A

\chapter{Append A}
\label{chap:anhangA}

\section{Teilanhang X}