% Autor: Manuel Lippert
% Physikalisches Praktikum

% Main-Datei für die Auswertung in TeX

% Struktur:
% Jedes Kapitel hat einen Input-File. Um Merge-Konflikte zu verhindern wird angeraten für jede 
% Datei eine eigene Tex Datei zu machen und sie im jeweiligen Kapitel zu importieren. Die in
% Input-Struktur dient zur besseren Übersicht und für mögliche Ordner. Die Zahlen vor den 
% Dateinamen dient zur Ordnung der einzelnen tex-Files nach Kapiteln


% Packages
\documentclass[paper=a4,bibliography=totoc,BCOR=10mm,twoside,numbers=noenddot,fontsize=11pt]{scrreprt}
\usepackage[ngerman]{babel}
\usepackage[T1]{fontenc}
\usepackage[latin1, utf8]{inputenc} %ä, ö, ü inbegriffen
\usepackage[babel,german=quotes]{csquotes} %For Quotes
\usepackage{lmodern}
\usepackage{graphicx}
\usepackage{nicefrac}
\usepackage{fancyvrb}
\usepackage{amsmath,amssymb,amstext}
\usepackage{siunitx}
\usepackage{url}
\usepackage{natbib}
\usepackage{microtype}
\usepackage[format=plain]{caption}
\usepackage{physics}
\usepackage{titleref} 

% Zusätzliche Packages
\usepackage{geometry} % Verändert Seitengeometrie
\usepackage{anyfontsize} % Alle Schriftgrößen möglich machen
\usepackage[table]{xcolor} % Farbliche Gestaltung Tabellen
\usepackage{ifthen} % Für kompliziertere tex-Files
\usepackage[absolute,overlay]{textpos} %Textboxen
\usepackage{amsfonts} % Schriftarten
\usepackage{xstring} % Stringoperationen
\usepackage{tikz} % Zeichnungen
\usepackage{pdfpages} % Import von pdfs (Protokolle)
\usepackage{hyperref} % Verlinkungen im Dokument

% Abschnittseinrückung und -abstand
% Die folgenden Zeilen sollen möglichst nicht verändert werden
\parindent 0.0cm
\parskip 0.8ex plus 0.5ex minus 0.5ex

% Anzahl und Größe von Gleitobjekten
% maximal 2 Objekte oben und unten
% erlaubt auch größere Bilder, welche die ganze Seite benötigen
% Die folgenden Zeilen sollen möglichst nicht verändert werden
\setcounter{bottomnumber}{2}
\setcounter{topnumber}{2}
\renewcommand{\bottomfraction}{1.}
\renewcommand{\topfraction}{1.}
\renewcommand{\textfraction}{0.}

%\sc und \bc veraltet. Daher: (20.09.2018)
\DeclareOldFontCommand{\sc}{\normalfont\scshape}{\@nomath\sc}
\DeclareOldFontCommand{\bf}{\normalfont\scshape}{\textbf}

% Verschiedenes
\pagestyle{headings}          % Der Seitenstil sollte möglichst nicht verändert werden
\graphicspath{{./Bilder/}}    % Der Pfad für die Abbildungen Abbildungen wird gesetzt
\VerbatimFootnotes            % \verb etc. auch in \footnotes mφglich

% Funktionen
\newcommand\tab[1][1cm]{\hspace*{#1}}
\newcommand{\vect}[1]{\boldsymbol{\mathbf{#1}}}
\newcolumntype{g}{>{\columncolor[rgb]{ .741,  .843,  .933}}l}

\begin{document}

    \nonfrenchspacing

    % 0. Kapitel Cover
    \input{00-Cover.tex}

    \thispagestyle{empty}
    \cleardoublepage
    \tableofcontents
    \cleardoublepage

    % 1. Kapitel Einleitung
    % 1. Einleitung

\chapter{Einleitung}
\label{chap:einleitung}

% Text

    % 2.Kapitel Fragen zur Vorbereitung
    % 2. Fragen zur Vorbereitung

\chapter{Fragen zur Vorbereitung}
\label{chap:fvz}

% Text

% Input der Teilaufgaben je nach Produktion der Nebendateien ohne Ordner
% Teilaufgabe X

\section{Teilaufgabe X}

% etc.

    % 3.Kapitel Protokoll
    \input{30-Protokoll.tex}

    % 4.Kapitel Versuchsauswertung
    % 4. Versuchsauswertung

\chapter{Auswertung und Diskussion}
\label{chap:versuchsauswertung}

% Text

% Input der Teilauswertung je nach Produktion der Nebendateien ohne Ordner
\input{41-TeilauswertungX.tex}

% etc.

    % 5.Kapitel Fazit
    % 5. Fazit

\chapter{Fazit}
\label{chap:fazit}


% Text

    % Anhang
    % 6. Anhang

\appendix

% Text

% Anhang A

\chapter{Append A}
\label{chap:anhangA}

\section{Teilanhang X}

    % Literatur
    \bibliographystyle{Auswertung.bst}
    \nocite{*}
    \bibliography{Auswertung.bib}

    % Abbildungsverzeichnis
    \listoffigures

\end{document}
